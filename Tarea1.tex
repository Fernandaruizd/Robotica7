\documentclass[12pt,a4paper,final]{article}
\usepackage[utf8]{inputenc}
\usepackage{amsmath}
\usepackage{amsfonts}
\usepackage{amssymb}
\usepackage{array}
\usepackage{listings}
\usepackage{float}
\usepackage[usenames, dvipsnames]{color}
\author{Fernanda Ruiz}
\title{Matrices-Robotica}
\begin{document}
\tableofcontents
\section{Introducción}
La matrices son ordenaciones de números dispuestos en filas y columnas. (Ayres, F. 1992) \cite{Ayres}
\subsection*{Ing. René Mario Montante Pardo}
Nació el 14 de septiembre de 1933. Es egresado de la Facultad de Ingeniería Mecánica y Eléctrica de la Universidad de Nuevo León. \newline Él desarrollo el Método Montante, que es un algoritmo para determinar soluciones de ecuaciones lineales, encontrar matrices de inversas, matrices de adjuntos y determinantes. El método es particular, porque se trabaja siempre con enteros, de tal manera que el resultado puede ser una fracción pero en ningún momento las cifras se redondean. La solución no es aproximada, sino exacta. \cite{Montante}

\section{Problema}
De la siguiente matriz A, obtener:
\begin{itemize}
\item det(A)
\item inv(A) \(\rightarrow\) Método Joint
\item inv(A) \(\rightarrow\) fórmula
\end{itemize}
A =
$
\left[\begin{matrix}
6 & -3 & 1 & -3 \\
7 & 8 & -7 &  9 \\
\textcolor{BurntOrange}{3}& \textcolor{BurntOrange}{-1} & \textcolor{BurntOrange}{-6} & \textcolor{BurntOrange}{0} \\
-2 & 4 & 2 & -6 \\
\end{matrix}\right]
$


\section{det(A)}
A = 3 
$
\left[\begin{matrix}
\textcolor{blue}{-3} &  1 &\textcolor{Orchid}{-3} \\
\textcolor{green}{8}  &\textcolor{blue}{-7} & \textcolor{yellow}{9} \\
\textcolor{CarnationPink}{4}& \textcolor{green}{2}&\textcolor{blue}{-6} \\
\textcolor{yellow}{-3} &  \textcolor{CarnationPink}{1} &\textcolor{green}{-3} \\
\textcolor{BurntOrange}{8} & -7 & \textcolor{CarnationPink}{9} \\
\end{matrix}\right]
$
- (-1) 
$
\left[\begin{matrix}
\textcolor{blue}{6} &  1 &\textcolor{Orchid}{-3} \\
\textcolor{green}{7}  &\textcolor{blue}{-7} & \textcolor{yellow}{9} \\
\textcolor{CarnationPink}{-2}& \textcolor{green}{2}&\textcolor{blue}{-6} \\
\textcolor{yellow}{6} &  \textcolor{CarnationPink}{1} &\textcolor{green}{-3} \\
\textcolor{BurntOrange}{7} & -7 & \textcolor{CarnationPink}{9} \\
\end{matrix}\right]
$
-6 
$
\left[\begin{matrix}
\textcolor{blue}{6} & -3 &\textcolor{Orchid}{-3} \\
\textcolor{green}{7}  &\textcolor{blue}{8} & \textcolor{yellow}{9} \\
\textcolor{CarnationPink}{-2}& \textcolor{green}{4}&\textcolor{blue}{-6} \\
\textcolor{yellow}{6} &  \textcolor{CarnationPink}{-3} &\textcolor{green}{-3} \\
\textcolor{BurntOrange}{7} & 8 & \textcolor{CarnationPink}{9} \\
\end{matrix}\right]
$

$
3\{-3\cdot7\cdot6-8\cdot2\cdot3+4\cdot1\cdot9- 3\cdot7\cdot4+9\cdot2\cdot3+6\cdot1\cdot8\} = 3\{-120\} = -360
$

$
+1\{6\cdot7\cdot6-7\cdot2\cdot3-2\cdot1\cdot9+ 3\cdot7\cdot2-9\cdot2\cdot6-6\cdot1\cdot7\} = 168
$

$
-6\{6\cdot8\cdot6-7\cdot4\cdot3+2\cdot3\cdot9- 3\cdot8\cdot2-9\cdot4\cdot6-6\cdot3\cdot7\} = -6\{-708\} = 4248
$

det(A) = -360 + 168 + 4248 = 4056

\section{Inversa}
\subsection{Fórmula}
Se obtiene la minor matrix:
\newline \newline \newline
$
\left[\begin{matrix}
\begin{vmatrix} %1
8 & -7 & 9\\
-1 & -6 & 0\\
4 & 2 & -6 \\
\end{vmatrix}
&
\begin{vmatrix}
7 & -7 & 9\\ %2
3 & -6 & 0\\
-2 & 2 & -6 \\
\end{vmatrix}
& 
\begin{vmatrix}
7 & 8 & 9\\  %3
3 & -1 & 0\\
-2 & 4 & -6 \\
\end{vmatrix}
&
\begin{vmatrix}
7 & 8 & -7\\  %4
3 & -1 & -7\\
-2 & 4 & 2 \\
\end{vmatrix}
\\ \\
\begin{vmatrix}
-3 & 1 & -3\\  %5
-1 & -6 & 0\\
4 & 2 & -6 \\
\end{vmatrix}
&
\begin{vmatrix}
6 & 1 & -3\\  %6
3 & -6 & 0\\
-2 & 2 & -6 \\
\end{vmatrix}
& 
\begin{vmatrix}
6 & -3 & -3\\  %7
3 & -1 & 0\\
-2 & 4 & -6 \\
\end{vmatrix}
&
\begin{vmatrix}
6 & -3 & 1\\  %8
3 & -1 & -7\\
-2 & 4 & 2 \\
\end{vmatrix}
\\ \\
\begin{vmatrix}
-3 & 1 & -3\\  %9
8 & -7 & 9\\
4 & 2 & -6 \\
\end{vmatrix}
&
\begin{vmatrix}
6 & 1 & -3\\ %10
7 & -7 & 9\\
-2 & 2 & -6 \\
\end{vmatrix}
& 
\begin{vmatrix}
6 & -3 & -3\\ %11
7 & 8 & 9\\
-2 & 4 & -6 \\
\end{vmatrix}
&
\begin{vmatrix}
6 & -3 & 1\\ %12
7 & 8 & -7\\
-2 & 4 & 2 \\
\end{vmatrix}
\\ \\
\begin{vmatrix}
-3 & 1 & -3\\ %13
8 & -7 & 9\\
-1 & -6 & 0 \\
\end{vmatrix}
&
\begin{vmatrix}
6 & 1 & -3\\ %14
7 & -7 & 9\\
3 & -6 & 0 \\
\end{vmatrix}
& 
\begin{vmatrix}
6 & -3 & -3\\ %15
7 & 8 & 9\\
3 & -1 & 0 \\
\end{vmatrix}
&
\begin{vmatrix}
6 & -3 & 1\\ %16
7 & 8 & -7\\
3 & -1 & -6 \\
\end{vmatrix}
\\
\end{matrix}\right]
$
\newline
\newline
Procedemos a obtener los determinates de cada uno:
\newline
\newline
$
\begin{vmatrix} %1
8 & -7 & 9\\
-1 & -6 & 0\\
4 & 2 & -6 \\
\end{vmatrix}
$
\newline
\newline
$
288 - 18 + 216 + 42 = 528 \newline
\newline
\begin{vmatrix} %2
7 & -7 & 9\\
3 & -6 & 0\\
-2 & 2 & -6 \\
\end{vmatrix} \newline
\newline
252 + 54 -108 - 126 = 72 \newline
\newline
\begin{vmatrix} %3
7 & 8 & 9\\
3 & -1 & 0\\
-2 & 4 & -6 \\ 
\end{vmatrix} \newline
\newline
42 + 108 - 18 + 144 = 276 \newline
\newline
\begin{vmatrix}
7 & 8 & -7\\  %4
3 & -1 & -7\\
-2 & 4 & 2 \\
\end{vmatrix} \newline
\newline
-14 -84 + 96 + 14 + 168 - 48 = 132
$
\newline
$
\newline
\begin{vmatrix}
-3 & 1 & -3\\  %5
-1 & -6 & 0\\
4 & 2 & -6 \\
\end{vmatrix} \newline
\newline
-108 + 6 - 72 - 6 = -180 \newline
\newline
\begin{vmatrix}
6 & 1 & -3\\  %6
3 & -6 & 0\\
-2 & 2 & -6 \\
\end{vmatrix} \newline
\newline
216 - 18 + 36 + 18 = 252 \newline
\newline
\begin{vmatrix}
6 & -3 & -3\\  %7
3 & -1 & 0\\
-2 & 4 & -6 \\
\end{vmatrix} \newline
\newline
36 - 36 + 6 - 54 = -48 \newline
\newline
\begin{vmatrix}
6 & -3 & 1\\  %8
3 & -1 & -7\\
-2 & 4 & 2 \\
\end{vmatrix} \newline
\newline
-12 + 12 - 36 - 2 + 144 + 18 = 124 \newline
$
\newline
$
\begin{vmatrix}
-3 & 1 & -3\\  %9
8 & -7 & 9\\
4 & 2 & -6 \\
\end{vmatrix} \newline
-126 - 48 + 36 - 84 + 54 + 48 = -120
\newline
\newline
\begin{vmatrix}
6 & 1 & -3\\ %10
7 & -7 & 9\\
-2 & 2 & -6 \\
\end{vmatrix} \newline
252 - 42 - 18 + 42 - 108 + 42 = 168
\newline
\newline
\begin{vmatrix}
6 & -3 & -3\\ %11
7 & 8 & 9\\
-2 & 4 & -6 \\
\end{vmatrix}
\newline
-288 - 84 + 54 - 48 - 216 -126 = -708
\newline
\newline
\begin{vmatrix}
6 & -3 & 1\\ %12
7 & 8 & -7\\
-2 & 4 & 2 \\
\end{vmatrix} \newline
96 + 28 - 42 + 16 + 168 + 42 = 308
$
\newline
\newline
$
\begin{vmatrix}
-3 & 1 & -3\\ %13
8 & -7 & 9\\
-1 & -6 & 0 \\
\end{vmatrix}
\newline
144 - 9 + 21 - 162 = - 6 \newline
\newline
\begin{vmatrix}
6 & 1 & -3\\ %14
7 & -7 & 9\\
3 & -6 & 0 \\
\end{vmatrix}
\newline
126 + 27 - 63 + 324 = 414 \newline
\newline
\begin{vmatrix}
6 & -3 & -3\\ %15
7 & 8 & 9\\
3 & -1 & 0 \\
\end{vmatrix}
\newline
21 - 81 + 72 + 54 = 66 \newline
\newline
\begin{vmatrix}
6 & -3 & 1\\ %16
7 & 8 & -7\\
3 & -1 & -6 \\
\end{vmatrix}
\newline
-288 - 7 + 63 - 24 - 42 - 126 = -424
$
\newline
\newline
A partir de ahí obtenemos la matriz de cofactores:
\newline
$
\left[\begin{matrix}
528 & -72 & 276 & -132 \\
180 & 252 & 48 & 124\\
-120 & -168 & -708 & -308 \\
6 & 414 & -66 & -424\\
\end{matrix}\right]
$
\newline
\newline
Sabemos que la adjunta es la transpuesta de la matriz de cofactores: \newline
$
\left[\begin{matrix}
528 & 180 & -120 & 6 \\
-72 & 252 & -168 & 414\\
276 & 48 & -708 & -66 \\
-132 & 124 & -308 & -424\\
\end{matrix}\right]
$
\newline \newline
Para obtener la inversa de la matriz A, solo hay que dividir la matriz adjunta entre el determinante: \newline \newline
$
\frac{1}{4056}
\left[\begin{matrix}
528 & 180 & -120 & 6 \\
-72 & 252 & -168 & 414\\
276 & 48 & -708 & -66 \\
-132 & 124 & -308 & -424\\
\end{matrix}\right]
$
\newline \newline
\newline \newline
$
inv(A)=
\left[\begin{matrix}
\frac{528}{4056} & \frac{180}{4056} & \frac{-120}{4056} & \frac{6}{4056} \\
\frac{-72}{4056} & \frac{252}{4056} & \frac{-168}{4056} & \frac{414}{4056}\\
\frac{276}{4056} & \frac{48}{4056} & \frac{-708}{4056} & \frac{-66}{4056} \\
\frac{-132}{4056} & \frac{124}{4056} & \frac{-308}{4056} & \frac{-424}{4056}\\
\end{matrix}\right]
$
\newline\newline
\newline \newline
$
inv(A)=
\left[\begin{matrix}
\frac{22}{169} & \frac{15}{338} & \frac{-5}{169} & \frac{1}{676} \\
\frac{-3}{169} & \frac{21}{338} & \frac{-7}{169} & \frac{69}{676}\\
\frac{23}{338} & \frac{2}{169} & \frac{-59}{338} & \frac{-11}{676} \\
\frac{-11}{338} & \frac{31}{1014} & \frac{-77}{1014} & \frac{-53}{507}\\
\end{matrix}\right]
$
\subsection{Montante}
$
inv(A)=
\left[\begin{matrix}
6 & -3 & 1 & -3 & 1 & 0 & 0 & 0 \\
7 & 8 & -7 &  9& 0 & 1 & 0 & 0\\
3 & -1 & -6 & 0& 0 & 0 & 1 & 0 \\
-2 & 4 & 2 & -6 & 0 & 0 & 0 & 1\\
\end{matrix}\right]
$
\newline \newline
\newline \newline
$
inv(A)=
\left[\begin{matrix}
6 & -3 & 1 & -3 & 1 & 0 & 0 & 0 \\
0 & 69 & -49 & 75 & -7 & 6 & 0 & 0\\
0 & 3 & -39 & 9 & -3 & 0 & 6 & 0 \\
0 & 18 & 14 & -42 & 2 & 0 & 0 & 6\\
\end{matrix}\right]
$
\newline \newline
\newline \newline
$
inv(A)=
\left[\begin{matrix}
69 & 0 & -13 & 3 & 8 & 3 & 0 & 0 \\
0 & 69 & -49 & 75 & -7 & 6 & 0 & 0\\
0 & 0 & -424 & 66 & -31 & -3 & 69 & 0 \\
0 & 0 & 308 & -708 & 44 & -18 & 0 & 69\\
\end{matrix}\right]
$
\newline \newline
\newline \newline
$
inv(A)=
\left[\begin{matrix}
-424 & 0 & 0 & -6 & -55 & -19 & 13 & 0 \\
0 & -424 & 0 & -414 & 21 & -39 & 49 & 0\\
0 & 0 & -424 & 66 & -31 & -3 & 69 & 0 \\
0 & 0 & 0 & 4056 & -132 & 124 & -308 & -424\\
\end{matrix}\right]
$
\newline \newline
\newline \newline
$
inv(A)=
\left[\begin{matrix}
4056 & 0 & 0 & 0 & 528 & 180 & 120 & 6 \\
0 & 4056 & 0 & 0 & -72 & 252 & -168 & 414\\
0 & 0 & 4056 & 0 & 276 & 48 & -708 & -66 \\
0 & 0 & 0 & 4056 & -132 & 124 & -308 & -424\\
\end{matrix}\right]
$
\newline \newline
\newline \newline
$
inv(A)=
\left[\begin{matrix}
1 & 0 & 0 & 0 &\frac{528}{4056} & \frac{180}{4056} & \frac{-120}{4056} & \frac{6}{4056} \\
0 & 1 & 0 & 0 &\frac{-72}{4056} & \frac{252}{4056} & \frac{-168}{4056} & \frac{414}{4056}\\
0 & 0 & 1 & 0 &\frac{276}{4056} & \frac{48}{4056} & \frac{-708}{4056} & \frac{-66}{4056} \\
0 & 0 & 0 & 1 &\frac{-132}{4056} & \frac{124}{4056} & \frac{-308}{4056} & \frac{-424}{4056}\\
\end{matrix}\right]
$
\newline\newline
\newline \newline
$
inv(A)=
\left[\begin{matrix}
1 & 0 & 0 & 0 &\frac{22}{169} & \frac{15}{338} & \frac{-5}{169} & \frac{1}{676} \\
0 & 1 & 0 & 0 &\frac{-3}{169} & \frac{21}{338} & \frac{-7}{169} & \frac{69}{676}\\
0 & 0 & 1 & 0 &\frac{23}{338} & \frac{2}{169} & \frac{-59}{338} & \frac{-11}{676} \\
0 & 0 & 0 & 1 &\frac{-11}{338} & \frac{31}{1014} & \frac{-77}{1014} & \frac{-53}{507}\\
\end{matrix}\right]
$
\section{Bibliografía}
\bibliographystyle{plain}
\begin{thebibliography}{9}
\bibitem{Ayres}
Ayres, F. (1992)
\newblock {\it Matrices} México: McGraw-Hill

\bibitem{Montante}
Zavala, N. (SF)
\newblock{\it Ing. René Mario Montante Pardo} Recuperado de {\it http://www.uanl.mx/emerito/ing-rene-mario-montante-pardo.html}
\end{thebibliography}
\end{document}